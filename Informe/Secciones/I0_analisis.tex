\HgraficarEPS{0.4}{cto_1}{Circuito completo.}{fig:cto1}

El Circuito comienza operando en modo continuo y gradualmente va pasando a modo discontinuo.

\HgraficarEPS{0.4}{asdas}{Esquema auxiliar.}{fig:cto_aux1}

Despreciando el efecto de los diodos en los transistores y los efectos parasitos que produce $C_3$ y $R_6$ ya que el tiempo de descarga es mucho menor al periodo total del circuito. 

\begin{equation}
	f = 20 \si{\kilo\Hz} , T = 50 \si{\micro}s 
\end{equation}

\begin{equation}
	\si{\tau}_{descarga} = C_3 \cdot R_6 = 0,5 \si{\micro}s << T
\end{equation}

Al ser este un circuito conocido se sabe que :

\begin{equation}
 	I_s = \frac{{I_{max} \cdot \si{\tau}_{on}}}{{2 \cdot T}} + \frac{{I_{max} \cdot \si{\tau}_{off}}}{{2 \cdot T}}
 	\label{ec:Is}
\end{equation}

 Suponiendo que $D= 0,5$

\begin{equation}
	\si{\tau}_{on} = D \cdot T = 25\si{\micro}s 
\end{equation}

Como el circuito tiende al modo discontinuo, se observa de la figura XX que $I_{min} = 0$, por lo tanto.
\begin{equation}
 	I_{max} = I_{min} + \frac{{V_{i} - V_{s}}}{L} \cdot \si{\tau}_{on} = \frac{{V_{i} - V_{s}}}{L} \cdot D \cdot T
\end{equation}

De la misma forma a partir de $I_{min}$ se obtiene $\si{\tau}_{off}$.

\begin{equation}
 	I_{min} = 0 \si{\ampere} =  I_{max} + \frac{V_{s} }{L} \cdot \si{\tau}_{off}  
\end{equation}

% \begin{equation}
%  	I_{min}  = \frac{{V_{i} - V_{s}}}{L} \cdot D \cdot T - \frac{V_{s} }{L} \cdot \si{\tau}_{off} = 0
% \end{equation}

Por lo tanto,

\begin{equation}
 	\si{\tau}_{off}  = \frac{{V_{i} - V_{s}}}{V_s} \cdot D \cdot T 
\end{equation}

Reemplazando los valores obtenidos en la ecuación \eqref{ec:Is}.

\begin{equation}
 	I_s = \frac{{V_{i} - V_{s}}}{L} \cdot \frac{(D \cdot T)^2 }{2 \cdot T} + \frac{{V_{i} - V_{s}}}{L} \cdot \frac{D \cdot T }{2 \cdot T} \cdot \frac{{V_{i} - V_{s}}}{V_s} \cdot D \cdot T
\end{equation}

Despejando me queda una ecuacion cuadratica para obtener $V_s$
\begin{equation}
	V_i - V_s + V^2_i -2 \cdot V_i \cdot V_s + V^2_s =  \frac{2 \cdot L \cdot I_s}{D^2 \cdot T}
\end{equation}

Se obtienen dos valores de $V_s$ al ser una fuente reductora:

\begin{equation}
 	 \boxed{V = 10,6\si{\volt} }
\end{equation}

Si se consideran las caidas de tensón en los transistores y en $R_5$
\begin{equation}
	I_{max} = \frac{V_i - V_s - 0,1 \si{ohm} \cdot I_{max} - 0,1 \si{ohm} \cdot I_{max} }{L} \cdot \si{\tau}_{on}
\end{equation}
Con $V_s$ ya calculado:
\begin{equation}
	 \boxed{I_{max} = 3,5 \si{\ampere}}
\end{equation}


Por lo tanto $V_{DSmax} = 0,35 \si{\volt}$ y $V_{R_1,...,R_5} = 0,35 \si{\volt}$

Esto desmuestra que fue correctamente despreciado.


\begin{equation}
	\boxed{\tau_{off} = 3,3 us = D_1 \cdot T}
\end{equation}

Entonces
\begin{equation}
 	\boxed{D_1  = 7/ 106 }
\end{equation}

Utilizando la formula de $V_s$ verifico el resultado.

\begin{equation}
 	V_s  = V_E \cdot \frac{D}{D + D_1} = 10,6\si{\volt}
\end{equation}
