\HgraficarEPS{0.4}{cto_1}{Circuito completo.}{fig:cto1}

El Circuito comienza operando en modo continuo y gradualmente va pasando a modo discontinuo.
\Gus{Ver bien y diganme si lo hicieron sikilar, Aca iria el grafico que estuve tratado de hacer es el asdas.cp o page2.svg}

\HgraficarEPS{0.4}{asdas}{Esquema auxiliar.}{fig:cto_aux1}

Despreciando el efecto de los diodos en los transistores y los efectos parasitos que produce $C_3$ y $R_6$ ya que el tiempo de descarga es mucho menor al periodo total del circuito. 


	\begin{equation}
		f = 20 \si{\kilo\Hz} , T = 0,5 \si{\micro}S << T
	\end{equation}


	\begin{equation}
		\si{\tau}_{descarga} = C_3 \cdot R_6 = 0,5 \si{\micro}S << T
	\end{equation}

Al ser este un circuito conocido se sabe que :

\begin{equation}
 	I_s = {I_{max} \cdot \si{\tau}}/{2 \cdot T}
\end{equation}
\Gus{esta cargado los resultados mañana termino de mirarlo}

 Suponiendo que $D= 0,5$

\begin{equation}
	\si{\tau} = D \cdot T
\end{equation}

% \Gus{mañana termino de cargarlo}
 
\begin{equation}
 	V = 10,6 V
\end{equation}



\begin{equation}
	I_{max} = 3,5 A
\end{equation}



\begin{equation}
 	V_{DSmax} = 0,35 V
\end{equation}



\begin{equation}
	tau_{ods} = 3,3 us
\end{equation}


\begin{equation}
 	Entonces  = 7/ 106 
\end{equation}
