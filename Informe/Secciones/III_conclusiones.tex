La principal diferencia entre los dos circuitos estudiados radica en la tensión de salida, en el primer caso siempre se cumple $V_S < V_E$, mientras que en el segundo circuito la tensión de salida siempre superará a la de entrada.
Otro parámetro importante a tener en cuenta es la tensión de rizado en la salida. En el primer caso se obtuvo un menor rizado que en el segundo. 

Por otra parte, se observa que ambos presentan transistores y diodos en paralelo. Esto permite obtener una mejor distribución de la corriente, haciendo que cada dispositivo maneje menos corriente y por ende haya menor caída de tensión en él.

Asimismo, poseen una red $RC$ denominada \textit{snub} en paralelo al diodo para el primer regulador y en paralelo al arreglo de transistores en el regulador \textit{Flyback}. La función del \textit{snub} es amortiguar la sobretensión que ocurre en el inductor ante las conmutaciones, evitando la destrucción de los dispositivos.


