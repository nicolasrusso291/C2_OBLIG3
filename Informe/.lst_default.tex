%% Configuración por default de los Listings de Latex

	\definecolor{myGreen}{rgb}{0,0.6,0}
	\definecolor{myGray}{rgb}{0.5,0.5,0.5}
	
	\lstset{
		inputencoding=utf8,
		extendedchars=true,              % lets you use non-ASCII characters; for 8-bits encodings only, does not work with UTF-8
%
		language=\myLanguage,		 	 %% the language of the code
		frame=single,	                 %% adds a frame around the code
		rulecolor=\color{Black},         % if not set, the frame-color may be changed on line-breaks within not-black text (e.g. comments (green here))
		basicstyle=\ttfamily\footnotesize,        % the size of the fonts that are used for the code
		backgroundcolor=\color{White},   % choose the background color; you must add \usepackage{color} or \usepackage{xcolor}
		keywordstyle=\color{Blue},       % keyword style
%		
		commentstyle=\itshape\color{myGreen},    % comment style
		stringstyle=\color{Purple},     % string literal style
		showstringspaces=false,          % underline spaces within strings only
		tabsize=4,	                   % sets default tabsize to 2 space
%
%
		breakatwhitespace=false,         % sets if automatic breaks should only happen at whitespace
		breaklines=true,                 %% sets automatic line breaking
		captionpos=t,                    %% sets the caption-position to top 
		deletekeywords={...},            % if you want to delete keywords from the given language
		escapeinside={\%*}{*)},          % if you want to add LaTeX within your code
		keepspaces=true,                 % keeps spaces in text, useful for keeping indentation of code (possibly needs columns=flexible)
		otherkeywords={*,...},           % if you want to add more keywords to the set
%
		numbers=left,                    %% where to put the line-numbers; possible values are (none, left, right)
		numbersep=5pt,                   %% how far the line-numbers are from the code
		numberstyle=\tiny\color{myGray}, % the style that is used for the line-numbers
		stepnumber=1,                    % the step between two line-numbers. If it's 1, each line will be numbered
%
		showspaces=false,                % show spaces everywhere adding particular underscores; it overrides 'showstringspaces'
		showtabs=false,                  % show tabs within strings adding particular underscores
		title={\protect\filename@parse{\lstname}\protect\filename@base\text{.}\protect\filename@ext}	 %% show the filename of files included with \lstinputlisting; also try caption instead of title
	}
