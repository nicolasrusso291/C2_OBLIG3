\input{.command.tex}
% En el siguiente archivo se configuran las variables del trabajo práctico
%% \providecommand es similar a \newcommnad, salvo que el primero ante un 
%% conflicto en la compilación, es ignorado.

% Al comienzo de un TP se debe modificar los argumentos de los comandos

\providecommand{\myTitle}{ACTIVIDAD 3}
\providecommand{\mySubtitle}{Fuentes conmutadas}

\providecommand{\mySubject}{Diseño de Circuitos Electrónicos (86.10)}
\providecommand{\myKeywords}{UBA, Ingeniería, C2}

\providecommand{\myAuthorSurname}{Alonso, Manso, Zuccolo}
\providecommand{\myTimePeriod}{Año 2018 - 2\textsuperscript{do} Cuatrimestre}

% No es necesario modificar este %%%%%%%%%%%%%%
\providecommand{\myHeaderLogo}{header_fiuba}
%%%%%%%%%%%%%%%%%%%%%%%%%%%%%%%%%%%%%%%%%%%%%%%%

% Si se utilizan listings, definir el lenguaje aquí
\providecommand{\myLanguage}{matlab}

% Crear los integrantes del TP con el comando \PutMember donde
%%		1) Apellido, Nombre
%%		2) Número de Padrón
%%		3) E-Mail
\providecommand{\MembersOnCover}[0]
{		
		\PutMember{Alonso, Gustavo Gabriel} {96119} {gustavoalon19@gmail.com}
		\PutMember{Manso, Juan} {96133} {juanmanso@gmail.com}
		\PutMember{Russo, Nicolas Emanuel} {93211} {nicolasrusso291@gmail.com}
		\PutMember{Zuccolo, Florencia} {96628} {florenciaz618@gmail.com}
}

\providecommand{\myGroupNumber}{10}

\Pagebreaktrue		% Setea si hay un salto de página en la carátula
\Indextrue
\Siunitxtrue			% Si quiero utilizar el paquete, \siunixtrue. Si no \siunixfalse
\Todonotestrue		% Habilita/Deshabilita las To-Do Notes y las funciones \unsure, \change, \info, \improvement y \thiswillnotshow.
\Listingsfalse
\Keywordsfalse
\Putgrouptrue		% Habilita/Deshabilita el \myGroup en los headers
\Videofalse
